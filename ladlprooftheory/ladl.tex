\documentclass{llncs}
\usepackage{amsmath}
\usepackage{amssymb}
\usepackage{comment}
\usepackage{hyperref}
\usepackage{longtable}
\usepackage{stmaryrd}
\newcommand{\interp}[1]{\llbracket #1 \rrbracket}
\newcommand{\maps}{\colon}
\renewcommand{\:}{\colon}
\newcommand{\FinSet}{\mathrm{FinSet}}
\newcommand{\Set}{\mathrm{Set}}
\newcommand{\Cat}{\mathrm{Cat}}
\newcommand{\Calc}{\mathrm{Calc}}
\newcommand{\Mon}{\mathrm{Mon}}
\newcommand{\BoolAlg}{\mathrm{BoolAlg}}
\renewcommand{\Form}{\mathrm{Form}}
\newcommand{\leftu}{\mathrm{left}}
\newcommand{\rightu}{\mathrm{right}}
\newcommand{\send}{\mathrm{send}}
\newcommand{\recv}{\mathrm{recv}}
\newcommand{\comm}{\mathrm{comm}}
\renewcommand{\quote}[1]{``#1"}
\newcommand{\deref}[1]{\mathrm{eval}(#1)}
\newcommand{\op}{\mathrm{op}}
\newcommand{\NN}{\mathbb{N}}
\newcommand{\pic}{$\pi$-calculus}

\makeatletter
\gdef\tshortstack{\@ifnextchar[\@tshortstack{\@tshortstack[c]}}
\gdef\@tshortstack[#1]{%
  \leavevmode
  \vtop\bgroup
    \baselineskip-\p@\lineskip 3\p@
    \let\mb@l\hss\let\mb@r\hss
    \expandafter\let\csname mb@#1\endcsname\relax
    \let\\\@stackcr
    \@ishortstack}
\makeatother

\title{Logic, resource, and reflection}
\author{
L.G. Meredith\inst{2}\\
\and
Michael Stay\inst{1}\\
}
\institute{
  {Pyrofex Corp.}\\
  \email{\fontsize{8}{8}\selectfont stay@pyrofex.net}\\
  \and
  {RChain Cooperative}\\
  \email{\fontsize{8}{8}\selectfont lgreg.meredith@rchain.coop}
}
\begin{document}
\maketitle
\begin{abstract}
\noindent
  We present an algorithm for deriving a spatial-behavioral type
  system and term assignment from a formal presentation of a
  interactive computational calculus. This turns out to identify a
  species of category which we offer an axiomatic characterization of
  interaction categories.
\end{abstract}

\section{Introduction and motivation}

In groundbreaking work, Abramsky, Gay, and Nagarajan put forward the
idea of interaction categories to give a categorical framework for
interactive models of computing. Indeed, interactive models of
computation are relatively new and offer distinct insights because
they include the computational environment as part of the
computation. Examples include the {\pic}, in fact, all the mobile
process calculi, as well as the lambda calculus. However, interaction
categories fail to say axiomatically what interaction is.

We put forward a simple, intuitive axiomatic characterization of
interaction and use this to derive not only a logic but a proof theory
and term assignment algorithm for all systems satisfying these
axioms. The resulting structure identifies a species of category and
we offer this as an axiomatic characterization of interaction
categories.

\subsection{Intuitions}

The key idea in this construction is to use both the evaluation
context of Meredith and Stay, and computational reflection, similar to
what is found in Meredith and Radestock, to build a proof theory with
a cut elimination that corresponds exactly with the notion of
computation embodied in any interactive rewrite system.

The requirement that the rewrite system is interactive means the left
hand side of every rewrite rule will necessarily be a term
constructor, say $\mathsf{K}$, taking at least two terms. That is, when
viewed as a piece of syntax it is at least a tuple of terms. Then, by
controlling evaluation with an evaluation context we can force a
distinction between a tensor and a cut, where the cut forms the redex
with an evaluation context supplied, and the tensor is denied the
evaluation context, effectively rendering it a data structure. Of
course, the data is inaccessible until there is a means to unpack it,
and a par term is constructed as the principal means to extract the
data from the tensor.

The use of the term constructor that forms a redex as a data structure
is the essential use of reflection. The lack of a reduction context
gives us the ability to suspend computation. Having suspended it, we
reify computation as data, and use par to unpack the data making up
the computation. Completing the circle, the tensor-par cut rule
constitutes the means to reflect computational data back into actual
computations.

It turns out that denying the evaluation context is not the only way
to suspend computation. We can also form contexts. In this case, we
only admit 1-holed contexts, ranged over by $\chi$. With contexts we can
define a notion of rely-guarantee, two notions actually (indicated by
the operations $(- \rhd -)$, and $(- \lhd -)$), as a redex term constructor
might not be commutative. .

\begin{eqnarray*}
\tau \rhd_{\mathsf{K}} \tau’ & \triangleq & \{ u \quad | \quad \exists t. u = \mathsf{!}\mathsf{K}[]r(t, []),\forall u’ : \tau. (\exists \rho : u@u’ \rightarrow v) \Rightarrow v:\tau’ \} \\
\tau’ \lhd_{\mathsf{K}} \tau & \triangleq & \{ t \quad | \quad \exists u. t = \mathsf{K}[]l([], u)\mathsf{!},\forall t’ : \tau. (\exists \rho: t@t’ \rightarrow v) \Rightarrow v : \tau’ \}
\end{eqnarray*}

In the sequel we drop the subscript on the triangles if it’s understood from
context. We use the notation $\chi @ t$ to mean the term formed by
substituting $t$ for hole in $\chi$. Thus, $\mathsf{K}(t, [])@u = \mathsf{K}(t, [])[u/[]] = \mathsf{K}(t,u)$ and similarly, $\mathsf{K}([], u)@t = \mathsf{K}([], u)[t/[]] = \mathsf{K}(t,u)$

To give some examples, the comm rule of rho-calculus is given by

\begin{equation*}
  \mathsf{for}( y \leftarrow x )P \quad | \quad x\mathsf{!}(Q) \rightarrow P\{ @Q/y \}
\end{equation*}

In this case K is parallel composition, and the revised, resource
constrained comm rule, looks like

\begin{equation*}
  R | \mathsf{for}( y \leftarrow x )P \quad | \quad x\mathsf{!}(Q) \rightarrow P\{ @Q/y \}
\end{equation*}

In the lambda calculus, $\beta$-reduction is given by

\begin{equation*}
  (\lambda x.M)N \rightarrow M\{ N/x \}
\end{equation*}  

$\mathsf{K}$ is application, and the revised, resource constrained $\beta$-reduction is given by

\begin{equation*}
  R((\lambda x.M)N) \rightarrow M\{ N/x \}
\end{equation*}    

In what follows we will focus on calculi that don’t employ binding
operators and so-called nominal phenomena. This is not to say that we
can’t handle nominal phenomena, just that the content is already
complex enough and we want to focus on the core ideas.


\subsection{Related work}
  

\section{Conclusion and future work}


\bibliographystyle{amsplain}
\bibliography{ladl}
\end{document}
