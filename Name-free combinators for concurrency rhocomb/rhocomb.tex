%%\documentclass{llncs}
\documentclass[submission,copyright,creativecommons]{eptcs}
\usepackage{tikz}
\usetikzlibrary{arrows,shapes}
\usepackage{mathpartir}
\usepackage{bigpage}
\usepackage{bcprules}
\usepackage{mathtools}
\usepackage{listings}
\usepackage{amsmath}
\usepackage{amssymb}
\usepackage{amsthm}
\usepackage{comment}
\usepackage{hyperref}
\usepackage{longtable}
\usepackage{stmaryrd}

\newcommand{\NYnote}[1]{\textbf{NY:#1}}
\newcommand{\new}{\mathsf{new}}
\newcommand{\interp}[1]{\llbracket #1 \rrbracket}
\newcommand{\maps}{\colon}
\newcommand{\Th}{\mathrm{Th}}
\newcommand{\Gph}{\mathrm{Gph}}
\newcommand{\FinSet}{\mathrm{FinSet}}
\newcommand{\FPGphCat}{\mathrm{FPGphCat}}
\newcommand{\Set}{\mathrm{Set}}
\newcommand{\Cat}{\mathrm{Cat}}
\newcommand{\Calc}{\mathrm{Calc}}
\newcommand{\Mon}{\mathrm{Mon}}
\newcommand{\BoolAlg}{\mathrm{BoolAlg}}
\renewcommand{\Form}{\mathrm{Form}}
\newcommand{\leftu}{\mathrm{left}}
\newcommand{\rightu}{\mathrm{right}}
\newcommand{\send}{\mathrm{send}}
\newcommand{\recv}{\mathrm{recv}}
\newcommand{\comm}{\mathrm{comm}}
\renewcommand{\quote}[1]{``#1"}
\newcommand{\deref}[1]{\mathrm{eval}(#1)}
\newcommand{\op}{\mathrm{op}}
\newcommand{\NN}{\mathbb{N}}
\newcommand{\pic}{$\pi$-calculus}
\newcommand{\ccomb}{$\mathsf{CC}$-calculus}

% Double brackets
\newcommand{\ldb}{[\![}
\newcommand{\rdb}{]\!]}
\newcommand{\ldrb}{(\!(}
\newcommand{\rdrb}{)\!)}
\newcommand{\lrbb}{(\!|}
\newcommand{\rrbb}{|\!)}
\newcommand{\lliftb}{\langle\!|}
\newcommand{\rliftb}{|\!\rangle}
%\newcommand{\plogp}{:\!-}
\newcommand{\plogp}{\leftarrow}
%\newcommand{\plogp}{\coloneq}
% \newcommand{\lpquote}{\langle}
% \newcommand{\rpquote}{\rangle}
% \newcommand{\lpquote}{\lceil}
% \newcommand{\rpquote}{\rceil}
\newcommand{\lpquote}{\ulcorner}
\newcommand{\rpquote}{\urcorner}
\newcommand{\wbbisim}{\stackrel{\centerdot}{\approx}} %weak barbed bisimilar

% SYNTAX
\newcommand{\id}[1]{\texttt{#1}}
\newcommand{\none}{\emptyset}
\newcommand{\eps}{\epsilon}
\newcommand{\set}[1]{\{#1\}}
\newcommand{\rep}[2]{\id{\{$#1$,$#2$\}}}
\newcommand{\elt}[2]{\id{$#1$[$#2$]}}
\newcommand{\infinity}{$\infty$}

\newcommand{\pzero}{\mathbin{0}}
\newcommand{\seq}{\mathbin{\id{,}}}
\newcommand{\all}{\mathbin{\id{\&}}}
\newcommand{\choice}{\mathbin{\id{|}}}
\newcommand{\altern}{\mathbin{\id{+}}}
\newcommand{\juxtap}{\mathbin{\id{|}}}
\newcommand{\concat}{\mathbin{.}}
\newcommand{\punify}{\mathbin{\id{:=:}}}
\newcommand{\fuse}{\mathbin{\id{=}}}
\newcommand{\scong}{\mathbin{\equiv}}
\newcommand{\nameeq}{\mathbin{\equiv_N}}
\newcommand{\alphaeq}{\mathbin{\equiv_{\alpha}}}
\newcommand{\names}[1]{\mathbin{\mathcal{N}(#1)}}
\newcommand{\freenames}[1]{\mathbin{\mathsf{FN}(#1)}}
\newcommand{\boundnames}[1]{\mathbin{\mathsf{BN}(#1)}}
%\newcommand{\lift}[2]{\texttt{lift} \; #1 \concat #2}
\newcommand{\binpar}[2]{#1 | #2}
\newcommand{\outputp}[2]{#1!(#2)}
\newcommand{\prefix}[3]{\mathsf{for}(#2 \leftarrow #1) #3}
\newcommand{\prefixst}[3]{\mathsf{for*}(#2 \leftarrow #1) #3}
\newcommand{\lift}[2]{#1 \lliftb #2 \rliftb}
\newcommand{\clift}[1]{\lliftb #1 \rliftb}
\newcommand{\quotep}[1]{\mathsf{@}#1}
\newcommand{\dropn}[1]{\mathsf{*}#1}
\newcommand{\procn}[1]{\stackrel{\vee}{x}}

\newcommand{\newp}[2]{(\newkw \; #1 ) #2}
\newcommand{\bangp}[1]{! #1}

\newcommand{\substp}[2]{\{ \quotep{#1} / \quotep{#2} \}}
\newcommand{\substn}[2]{\{ #1 / #2 \}}

\newcommand{\psubstp}[2]{\widehat{\substp{#1}{#2}}}
\newcommand{\psubstn}[2]{\widehat{\substn{#1}{#2}}}

\newcommand{\applyp}[2]{#1 \langle #2 \rangle}
\newcommand{\absp}[2]{( #1 ) #2}
\newcommand{\annihilate}[1]{#1^{\times}}
\newcommand{\dualize}[1]{#1^{\bullet}}

\newcommand{\transitions}[3]{\mathbin{#1 \stackrel{#2}{\longrightarrow} #3}}
\newcommand{\meaningof}[1]{\ldb #1 \rdb}
\newcommand{\pmeaningof}[1]{\ldb #1 \rdb}
\newcommand{\nmeaningof}[1]{\lrbb #1 \rrbb}

\newcommand{\Proc}{\mathsf{Proc}}
\newcommand{\QProc}{\quotep{\mathsf{Proc}}}

\newcommand{\bc}{\mathbin{\mathbf{::=}}}
\newcommand{\bm}{\mathbin{\mathbf\mid}}

\newcommand{\rel}[1]{\;{\mathcal #1}\;} %relation
\newcommand{\red}{\rightarrow}
\newcommand{\wred}{\Rightarrow}
\newcommand{\redhat}{\hat{\longrightarrow}}
\newcommand{\lred}[1]{\stackrel{#1}{\longrightarrow}} %transitions
\newcommand{\wlred}[1]{\stackrel{#1}{\Longrightarrow}}
\newcommand{\vect}[1]{\stackrel{\rightharpoonup}{#1}}

\newcommand{\rhoc}{$\rho$-calculus}
\newcommand{\rhocc}{$\rho$$\mathsf{CC}$}

\theoremstyle{definition}
\newtheorem{definition}{Definition}
\newtheorem{theorem}{Theorem}
%% \newtheorem{proof}{Proof}
\theoremstyle{remark}
\newtheorem{remark}{Remark}
\theoremstyle{remark}
\newtheorem{example}{Example}
\newtheorem{conjecture}{Conjecture}

\makeatletter
\gdef\tshortstack{\@ifnextchar[\@tshortstack{\@tshortstack[c]}}
\gdef\@tshortstack[#1]{%
  \leavevmode
  \vtop\bgroup
    \baselineskip-\p@\lineskip 3\p@
    \let\mb@l\hss\let\mb@r\hss
    \expandafter\let\csname mb@#1\endcsname\relax
    \let\\\@stackcr
    \@ishortstack}
\makeatother

\title{Name-free combinators for concurrency}
%% \author{
%% L.G. Meredith\inst{1}\\
%% \and
%% Michael Stay\inst{2}\\
%% }
%% \institute{
%%   {RChain Cooperative}\\
%%   \email{\fontsize{8}{8}\selectfont greg@rchain.coop}
%%   \and
%%   {Pyrofex Corp.}\\
%%   \email{\fontsize{8}{8}\selectfont stay@pyrofex.net}\\
%% }
\author{L.G. Meredith
\institute{RChain Cooperative\\ Washington State}
\email{lgreg.meredith@rchain.coop}
\and
Michael Stay
\institute{Pyrofex Corp.\\Utah, USA}
\email{\quad stay@pyrofex.net}
}
\def\titlerunning{Name-free combinators for concurrency}
\def\authorrunning{L.G. Meredith, M. Stay}
\def\publicationstatus{submitted to: PLACES 2020}
\begin{document}
\maketitle
\begin{abstract}
\noindent
  Developing an entirely name-free account of mobile concurrency has
  long been a goal of the concurrency semantics community, but to date
  has been somewhat elusive. However, it turns out that there are two
  partial solutions that may be combined. In particular, Honda and
  Yoshida eliminate the use of abstraction in input-guarded processes
  \cite{DBLP:conf/popl/HondaY94,DBLP:journals/tcs/Yoshida02}, but
  still make heavy use of the $\mathsf{new}$ operator. Meanwhile
  Meredith and Radestock \cite{DBLP:journals/entcs/MeredithR05}
  eliminate the $\mathsf{new}$ operator, but make essential use of
  input-guard abstraction. The two solutions are compatible and may be
  composed, yielding the first entirely name-free account of mobile
  concurrency that does not suffer a centralized resource ensuring
  global uniqueness of names.
\end{abstract}

\section{Introduction}

Our purpose in reporting these results is singular. We have developed
an algorithm, the logic as a distributive law algorithm (aka LADL),
that calculates a type system for a given model of computation. We
want to test the limits of our algorithm. In particular, we would like
to apply the algorithm to a model of computation that enjoys the most
advanced features of computation today. For us, this means mobile
concurrency, as evinced in models of computation like the {\pic} or
the {\rhoc}
\cite{DBLP:conf/concur/Milner92,DBLP:journals/entcs/MeredithR05}.
However, all such models currently proposed make heavy use of nominal
phenomena. Even Honda and Yoshida's combinatory model
\cite{DBLP:conf/popl/HondaY94,DBLP:journals/tcs/Yoshida02}
makes heavy use of nominal phenomena. While our algorithm can handle
nominal phenomena, it seriously complicates the presentation of the
core features of the algorithm, which we hope to illustrate in a
series of upcoming papers. So, we would like to test our algorithm on
a version of mobile concurrency that makes no use of nominal
phenomena.

It turns out that developing an entirely name-free account of mobile
concurrency has long been a goal of the concurrency semantics
community, but to date has been somewhat elusive. However, it turns
out that there are two partial solutions that may be combined. In
particular, Honda and Yoshida eliminate the use of abstraction in
input-guarded processes, but still make heavy use of the
$\mathsf{new}$ operator. Meanwhile Meredith and Radestock
\cite{DBLP:journals/entcs/MeredithR05} eliminate the $\mathsf{new}$
operator, but make essential use of input-guard abstraction. The two
solutions are compatible and may be composed, yielding the first
entirely name-free account of mobile concurrency, that does not suffer
a centralized resource ensuring global uniqueness of names. This model
of computation is amenable to the most elementary form of our
algorithm for generating type systems, which already counts as an
illustration of its utility!

However, it is useful to consider the larger context motivating these
results. It is remarkable that a single feature,
i.e. \emph{reflection}, suffices to produce the first such entirely
combinatory model of mobile concurrency. Reflection has been the ugly
step child of theoretical computer science, resisting compelling type
theoretic analysis. This is one of the reasons why language designs
based on typed $\lambda$-calculi have been so slow to adopt reflection
as a feature by comparison to other language designs. On the other
hand, reflection and meta-programming features more generally are
already part of a number of mainstream languages. Java, C\#, Scala,
even the Haskell and OCaml communities have seen growing interest in
these features with efforts like template Haskell and MetaOCaml,
respectively. In large measure this has to do with the practical
realities of programming at industrial scale. Such efforts require the
leverage of computer programs to write computer programs. Thus,
meta-programming features are simply a practical necessity.

It is worth noting that there are deep foundational questions at play
here. Many term calculi, like $\lambda$-calculus or {\pic}, involve
binders for names, and the mathematics of bound variable names is
subtle. Sch\"onfinkel introduced the $\mathsf{SKI}$ combinator
calculus in 1924 to clarify the role of quantified variables in
intuitionistic logic by eliminating them \cite{finkel}; Curry
developed Sch\"onfinkel's ideas much further. These subtleties are not
merely theoretical, but represent a real practical challenge in the
design of programming languages. In fact binding is one of the key
features of the PoPLMark Challenge \cite{PoPLMark}. Certainly, the work by Jamie Gabbay and
Andrew Pitts \cite{DBLP:journals/fac/GabbayP02} and others
\cite{DBLP:journals/jcss/Clouston14} on nominal set theory has put the
study of bound names and substitution on a much nicer foundation that
can be shown to extend to practical implementations. However, it
introduces an intriguing conundrum.

Fraenkel-Mostowski set theory (which we abbreivate to FM set theory)
must suffer the existence of \emph{infinite} supply of ``atoms''
\cite{DBLP:journals/fac/GabbayP02}. In Chaitin's analysis of the risk
present in a theory \cite{chaitin1999unknowable}, ``atoms'' constitute
a source of risk. The question is: where do these atoms come from?
The answer has both theoretical and practical implications. On the
practical side, infinite sets of ``atomic'' entities, i.e. entities
with no internal structure, are not realizable on modern
computers. Modern computers fundamentally rely on sets of elements
with effective internal structure to provide the kind of compression
necessary to produce or compute with infinite sets. The natural
numbers is a prime example. Because of their very regular internal
structure the entire set can be represented by a single recursive
equation. Potentially then, the natural numbers or some other
effectively representable set could provide the source of atoms used
in an FM-set theoretic account of binding in practical
implementations.

However, this raises a new question, one of circularity. In order to
represent and compute these effectively representable sets that will
be supplied as atoms, a notion of computation must already be in
place. If that notion of computation relies on a notion of binding,
then not a lot progress has been made! As in
\cite{DBLP:journals/entcs/MeredithR05} we argue that this circularity,
instead of being an obstacle to overcome, might be a clue to an
alternative approach to binding phenomena. What better place to
look for the source of ``atoms'' than in the reification of the theory
of computation requiring them? This answer ties together
two important computational phenomena that have not normally been
considered as related. More explicitly, we extend the argument made by
Meredith and Radestock that reflection suffices to provide the
``atoms'' used in the {\pic} as channels to produce the first
name-free set of combinators that enjoys a full and faithful
interpretation of the calculus.

To be clear, the model eliminates both ordinary abstraction as well as
binders for fresh names. For example, the {\pic}
\cite{milner91polyadicpi} has two binders for names: the $\new$
operator, which introduces a new name into scope, and the input
prefix, which introduces a name for labeling locations for
substitution.  Honda and Yoshida
\cite{DBLP:conf/popl/HondaY94,DBLP:journals/tcs/Yoshida02} describe an
elimination algorithm that gets rid of input prefixes which
corresponds in many respects to the elimination of lambda abstraction;
but their combinators (referred to in the sequel as the {\ccomb})
still fundamentally depend on the $\new$ operator.  In complementary
work, Meredith and Radestock \cite{DBLP:journals/entcs/MeredithR05}
introduce reflective operators into a higher-order {\pic} and
implement $\new$ and replication in terms of reflection.  This work
presents a fusion of those ideas: a name-free concurrent combinator
calculus into which the {\ccomb} have a faithful embedding.

It is worth mentioning that these results mean that it is possible to
extend Honda and Yoshida's original minimality and separation results
\cite{DBLP:journals/tcs/Yoshida02}. Specifically, this calculus uses
neither the $\mathsf{new}$ operator nor replication, but only the
original combinators together with a single combinator for reflection,
and yet captures the full power of the {\pic}. Thus, it is possible
extend the original results by considering combinator sets that
include or exclude the reflection combinator. In light of these
considerations we felt it worthwhile to document the model prior to
reporting on the LADL algorithm.

%% \paragraph*{Role of compositionality in theories of concurrency} 
%% Since the {\pic} is not a closed theory, but one dependent on a set of
%% names, we can imagine supplying the theory itself as a set of names
%% and regard the {\rhoc} \cite{DBLP:journals/entcs/MeredithR05} as a
%% kind of fixed point. Symbolically, the {\rhoc} is the fixed point of
%% the equation: $\mathsf{R} = \Pi[\quotep{\mathsf{R}}]$, where
%% $\Pi[X]$ denotes the theory of processes generated from the names
%% given by $X$, and $\quotep{\Pi[X]}$ denote the quoted forms of the
%% processes so generated. This equation captures exactly the kind of
%% design level thinking compositionality should foster, and it is
%% actually surprising that this particular solution for a name-free
%% version of a concurrent calculus was not found sooner. In point of
%% fact, the initial implementations of {\rhoc} in OCaml, Haskell, and
%% Scala all used this very fixed point formulation to define the syntax
%% for the {\rhoc}. The fact that the code type-checks and that the types
%% are inhabited provides some assurance of the soundness of the
%% construction.

%% The combinator versions take compositionality a step further by
%% removing even more of the syntax, effectively arriving at a variant of
%% an applicative algebra over a small handfull of combinators to account
%% for all binding and mobility phenomena. Further still, the soundness
%% of the semantics makes essential use of compositionality because we
%% are effectively composing the {\ccomb}' original semantics with the
%% reflective account of name construction and deconstruction; thus
%% illustrating that compositionality is not just for design-level
%% thinking, but provides powerful compression of proofs.

\paragraph*{Outline}
To be fully self-contained this paper needs to present four different
calculi and two different encodings: the original {\pic}, {\ccomb},
and the encoding from the term calculus to the combinator; the
{\rhoc}, and the encoding from the {\pic} to the {\rhoc} and the new
reflective combinator calculus, and the encoding from the term
calculus to the reflective combinator calculus. Such a manifest
provides all the technical inventory to illustrate how the two
encoding techniques, prefix elimination and $\new$ elimination,
combine and how the encoding from the {\pic} can be constructed by
composing the encoding into the {\rhoc} with the encoding into the
reflective combinator calculus. We have provided the complete
manifest, but have pushed the presentation of the {\rhoc} and the
{\pic{}} and the encoding of the latter into the former into one
Appendix~\ref{appendix:pic2rhoc}, and the details of the {\ccomb} into
another Appendix~\ref{appendix:ccomb}. This organization allows us to focus on
the newer results of a name-free combinator calculus for mobile
concurrency.


\section{A reflective higher-order concurrent combinator calculus}

\subsection{The {\ccomb}}

We summarize the original {\ccomb} \cite{DBLP:journals/tcs/Yoshida02} below.

\begin{mathpar}
  \inferrule* [lab=atom] {} { P \bc 0 \;|\; \mathsf{m}(a,b) \;|\; \mathsf{d}(a,b,c) \;|\; \mathsf{k}(a) \;|\; \mathsf{fw}(a,b) \;|\; \mathsf{b}_{\mathsf{r}}(a,b) \;|\; \mathsf{b}_{\mathsf{l}}(a,b) \;|\; \mathsf{s}(a,b,c) }
  \and
  \inferrule* [lab=process] {} {\bm \; (\new\; a)P \;|\; P|P \;|\; \mathsf{*}P}
\end{mathpar}

We write $\mathsf{c};\mathsf{c'};\ldots;$ to denote these
agents. $\mathsf{m}(a,b)$ (message) carries $a$ name $b$ to name $a$,
$\mathsf{d}$ (duplicator) distributes a message to two locations,
$\mathsf{fw}$ (forwarder) forwards a message (thus linking two
locations), $\mathsf{k}$ (killer) kills a message, while
$\mathsf{b}_{\mathsf{r}}$ (right binder), $\mathsf{b}_{\mathsf{l}}$
(left binder) and $\mathsf{s}$ (synchroniser) generate new links. In
particular $\mathsf{b}_{\mathsf{r}}$ and $\mathsf{b}_{\mathsf{l}}$
represent two different ways of binding names – in
$\mathsf{b}_{\mathsf{r}}$ one uses the received name for output, while
in $\mathsf{b}_{\mathsf{l}}$ one uses it for input. In contrast,
$\mathsf{s}$ is used for pure synchronisation without value passing,
which is indeed necessary in interaction scenarios.

These intuitions are formalized in the following rewrite rules.
\[\begin{array}{rl}
  \mathsf{d}(a,b,c) | \mathsf{m}(a,x) & \red \mathsf{m}(b,x) | \mathsf{m}(c,x) \\
  \mathsf{k}(a) | \mathsf{m}(a,x) & \red 0 \\
  \mathsf{fw}(a,b) | \mathsf{m}(a,x) & \red \mathsf{m}(b,x) \\
\end{array} \quad \quad
\begin{array}{rl}
  \mathsf{b}_{\mathsf{r}}(a,b) | \mathsf{m}(a,x) & \red \mathsf{fw}(b,x) \\
  \mathsf{b}_{\mathsf{l}}(a,b) | \mathsf{m}(a,x) & \red \mathsf{fw}(x,b) \\
  \mathsf{s}(a,b,c) | \mathsf{m}(a,x) & \red \mathsf{fw}(b,c)
\end{array}\]

Together with the usual rules for reduction in context.
As in the {\pic}, the $\new$ operator is a binding operator for names,
so {\ccomb} also has a notion of free and bound names. The
details are provided in Appendix~\ref{appendix:ccomb} to make the paper more
self-contained.

\subsection{Reflective higher-order (RHO) combinator calculus}
The {\pic} is not a closed theory, but rather a theory dependent upon
some theory of names. Taking an operational view, one may think of the
{\pic} as a procedure that when handed a theory of names provides a
theory of processes that communicate over those names. This openness
of the theory has been exploited in {\pic} implementations like the
execution engine in Microsoft's Biztalk \cite{biztalk}, where an
ancillary binding language provides a means of specifying a `theory'
of names: {\em e.g.}, names may be TCP/IP ports, or URLs, or object
references, {\em etc.}  But foundationally, one might ask if there is
a closed theory of processes, {\em i.e.} one in which the theory of
names arises from and is wholly determined by the theory of
processes. The work in \cite{DBLP:journals/entcs/MeredithR05}
demonstrates that this is not only possible, but results in a calculus
that enjoys both the features of concurrency and meta-programming. The
key idea is to provide the ability to quote processes, effectively
reifying them as names, and to unquote them, effectively reflecting
names back as processes.

This section applies the quotation technique developed in
\cite{DBLP:journals/entcs/MeredithR05} to {\ccomb}. We remove new
names and replication, introduce quoting/unquoting operators. Notice
that this effectively allows processes in the second argument of a
send, because the name in the second argument is `merely' a quoted
process. We introduce an extra rewrite
governing the interaction between sending and unquoting.

\begin{mathpar}
  \inferrule* [lab=atom] {} { P \bc 0 \;|\; \mathsf{m}(a,\quotep{P}) \;|\; \mathsf{d}(a,b,c) \;|\; \mathsf{k}(a) \;|\; \mathsf{fw}(a,b) \;|\; \mathsf{b}_{\mathsf{r}}(a,b) \;|\; \mathsf{b}_{\mathsf{l}}(a,b) \;|\; \mathsf{s}(a,b,c) }
  \and
  \inferrule* [lab=process] {} {\bm \; *a \;|\; P|P}
  \and
  \inferrule* [lab=nominal] {} {a \bc \quotep{P}}
\end{mathpar}

\noindent Rewrite rules
\[\begin{array}{rl}
\mathsf{d}(a,b,c) | \mathsf{m}(a,\quotep{P}) & \red \mathsf{m}(b,\quotep{P}) | \mathsf{m}(c,\quotep{P}) \\
\mathsf{k}(a) | \mathsf{m}(a,\quotep{P}) & \red 0 \\
\mathsf{fw}(a,b) | \mathsf{m}(a,\quotep{P}) & \red \mathsf{m}(b,\quotep{P}) \\
\mathsf{b}_{\mathsf{r}}(a,b) | \mathsf{m}(a,\quotep{P}) & \red \mathsf{fw}(b,\quotep{P}) \\  
\end{array} \quad \quad
\begin{array}{rl}
  \mathsf{b}_{\mathsf{l}}(a,b) | \mathsf{m}(a,\quotep{P}) & \red \mathsf{fw}(\quotep{P},b) \\
  \mathsf{s}(a,b,c) | \mathsf{m}(a,\quotep{P}) & \red \mathsf{fw}(b,c) \\
  *(a) | \mathsf{m}(a,\quotep{P}) & \red P
\end{array}\]

\begin{definition}
  The {\em structural congruence} $\equiv$
  between processes \cite{SangiorgiWalker} is the least congruence
  satisfying the commutative monoid laws
  (associativity, commutativity and $\pzero$ as identity) for parallel
  composition $|$ and $*(@(P)) \equiv P$.
\end{definition}

Note that alpha equivalence is no longer part of structural
congruence.  While there is a faithful embedding of {\ccomb} into
concurrent combinators of {\rhoc} (defined below as {\rhocc}),
{\rhocc} can see
the internal structure of names and distinguish them.

\subsubsection{Implementing replication with reflection}

% D_x = x?(y).(x<*y> | *y)
% 
%       x#y.(x<*y> | *y)
% 1 => vc1c2.(d(xc1c2) | c1#y.m(x*y) | c2#y.*y)
% 7 => vc1c2.(d(xc1c2) | fw(c1x) | c2#y.*y)
% * => vc1c2.(d(xc1c2) | fw(c1x) | *c2)
% 
% 
% 
% vc1c2.(d(xc1c2) | fw(c1x) | *c2) | m(xP)
% => vc1c2.(d(xc1c2) | fw(c1x) | *c2 | m(xP))
% => vc1c2.(d(xc1c2) | m(xP) | fw(c1x) | *c2)
% => vc1c2.(m(c1P) | m(c2P) | fw(c1x) | *c2)
% => vc1c2.(m(xP) | m(c2P) | *c2)
% => vc1c2.(m(xP) | P)
% 
% !_x P = m(x(D_x | P)) | D_x
% = vc1c2.(d(xc1c2) | fw(c1x) | *c2 | m(x(D_x | P)))
% = vc1c2.(m(c1(D_x | P)) | m(c2(D_x | P)) | fw(c1x) | *c2 | m(x(D_x | P)))
% = vc1c2.(m(x(D_x | P)) | m(c2(D_x | P)) | *c2)
% = vc1c2.(m(x(D_x | P)) | D_x | P)
% = m(x(D_x | P)) | D_x | P
As mentioned before, it is known that replication (and hence
recursion) can be implemented in a higher-order process algebra
\cite{SangiorgiWalker}. As our first example of calculation with the
machinery thus far presented we give the construction explicitly in
the RHO combinator calculus, abbreviated {\rhocc} in the sequel.
\begin{definition}[Replication]
  \label{replication}
  $D(x,v,w) := (\mathsf{d}(x,v,w) | \mathsf{fw}(v,x) | {*}(w))$
\end{definition}
By the above definition, we have:
\[\begin{array}{rl}
  \mathsf{*}_{(x,v,w)} P &= \mathsf{m}(x,\quotep{(D(x,v,w) \; |\; P)}) \; |\; D(x,v,w) \\
        &= \mathsf{d}(x,v,w) \; |\; \mathsf{fw}(v,x) \; |\; {*}(w) \; |\; \mathsf{m}(x,\quotep{(D(x,v,w) \; |\; P)}) \\
        &\red \mathsf{m}(v,\quotep{(D(x,v,w) \; |\; P)}) \; |\; \mathsf{m}(w,\quotep{(D(x,v,w) \; |\; P)}) \; |\; \mathsf{fw}(v,x) \; |\; {*}(w) \; |\; \mathsf{m}(x,\quotep{(D(x,v,w) \; |\; P)}) \\
        &\red \mathsf{m}(x,\quotep{(D(x,v,w) \; |\; P)}) \; |\; \mathsf{m}(w,\quotep{(D(x,v,w) \; |\; P)}) \; |\; {*}(w) \\
        &\red \mathsf{m}(x,\quotep{(D(x,v,w) \; |\; P)}) \; |\; D(x,v,w) \; |\; P \\
        & = \; \mathsf{*}_{(x,v,w)} P \; |\; P
\end{array}\]
This encoding, as an implementation, runs away, unfolding
$\mathsf{*}P$ eagerly. It is possible to obtain a lazier
replication operator restricted to the embedding of
input-guarded $\pi$ processes. The reader familiar with the
$\lambda$-calculus will have noticed the similarity between $D$ and
the ``fixed point'' combinator $Y$.

\subsubsection{Implementing new names with reflection}

Here we provide an encoding of {\ccomb} into the {\rhocc}. Since all
names are global in the {\rhocc}, we encounter a small complication in
the treatment of free names at the outset. There are several ways to
handle this.  One is to insist that the translation be handed a closed
program, one in which all names are bound either by input or by
restriction, but this feels inelegant. Another is to provide an
environment, $r : \mathcal{N}_{\mathsf{CC}} \rightarrow \QProc$, for
mapping the free names in a {\ccomb} process into names in the
{\rhocc}. Maintaining the updates to the environment, however,
obscures the simplicity of the translation. We adopt a third
alternative.

To hammer home the point that {\ccomb} is parameterized in a theory of
names, we instantiate the calculus with the names of the
{\rhocc}. This is no different than instantiating the {\ccomb} using
the natural numbers, or the set of URLs as the set of names. Just as
there is no connection between the structure of these kinds of names
and the structure of processes in the {\pic}, there is no connection
between the processes quoted in the names used by the theory and the
processes generated by the theory, and we exploit this fact.

Let $\QProc$ be set of names in the {\rhocc}, $\Proc$ be the set of
terms of the {\rhocc}, and $\Proc_{\mathsf{CC}}$ be the set of terms
of the {\ccomb} built using $\QProc$ \emph{as the names}. The
translation will be given as a function \[\meaningof{-}_2( -,
- ) : \Proc_{\mathsf{CC}} \times \QProc {\times} \QProc \red \Proc.\]
The guiding intuition is that we construct alongside the process a
distributed memory allocator, the process' access to which is mediated
through the second argument to the function, called $p$ below. The
first argument, called $n$ below, determines the shape of the memory
for the given allocator.

Since the {\ccomb} is parametric in a set of names, we can choose
$\QProc$ for that set.  Given a process $P$ in {\ccomb}, we pick names
$n$ and $p$ in the {\rhoc} such that $n \neq p$ and both are distinct
from the free names of $P$.  Then we define the mapping with two name
constructors $x^r$ and $x^l$ as:
\begin{equation*}
  \meaningof{P} = \meaningof{P}_2(n, p)
  \quad \mbox{with} \quad 
   x^l := \quotep{\mathsf{b}_{\mathsf{l}}(x,\mathsf{m}(x,*x))}, \ 
   x^r := \quotep{\mathsf{b}_{\mathsf{r}}(x,\mathsf{m}(x,*x))}
\end{equation*}
%\begin{tabular}{cc}
%  $x^l := \quotep{\mathsf{b}_{\mathsf{l}}(x,\mathsf{m}(x,*x))}$ & $x^r := \quotep{\mathsf{b}_{\mathsf{r}}(x,\mathsf{m}(x,*x))}$
%\end{tabular}
Note that by construction, $\quotep{P}$ cannot occur as a name in $P$
and hence any name derived from a process that is built using
$\quotep{P}$ cannot occur in $P$. Thus, the effect of the superscripts
$l$ and $r$ on a name $x$ is to construct a name that is guaranteed to
be fresh with respect to the free names of the process being
interpreted. More generally, mentioning a name, say $n$, in the
constructor of a another name, say $n'$, guarantees distinction
between $n$ and $n'$; likewise, mentioning a process, say $P$, in the
constructor of a name $n$ guarantees that $n$ is fresh in $P$. The
particular choices of combinators used in the name
constructors are irrelevant for the purposes of freshness. We
make heavy use of this fact in our interpretation of prefix
elimination.

The interpretation function $\meaningof{-}_2(n, p)$ is straightforward
for all but replication and $\mathsf{new}$.
\[\begin{array}{ccc}
\meaningof{\pzero}_2 (n,p) = \pzero & \meaningof{\emph{c}(\vect{a})}_2 (n,p) = \emph{c}(\vect{a}) & \meaningof{P \juxtap Q}_2 (n,p)  =  \meaningof{P}_2 (n^l, p^l) \juxtap \meaningof{Q}_2 (n^r, p^r)  
\end{array}\]
These latter two forms require extra care. We define them in terms of
a prefix form and then use a version of Honda and Yoshida's prefix elimination
to remove the prefix.
\begin{eqnarray*}
    \meaningof{\mathsf{*} P}_2 (n,p)
          & = & \binpar{\mathsf{m}(x, \quotep{\meaningof{P}_3(n^r,p^r)})}
                  {\binpar{D(x,v,w)}
                    {\binpar{\mathsf{m}(n^r, *n^l)}{\mathsf{m}(n^r, *n^l)}}} \\
                  & & \mbox{where } 
                      x = @(\meaningof{P}_2(n,p))^{ll}, 
                      v = @(\meaningof{P}_2(n,p))^{lr}, \mbox{ and }
                      w = @(\meaningof{P}_2(n,p))^{rr} \\
    \meaningof{(\new \; x ) P}_2 (n, p) 
          & = & 
         \binpar{\meaningof{\prefix{p}{x}{{\meaningof{P}_2 ( n^l, p^l )}}}_4(n, p)}{\mathsf{m}(p, n)}
\end{eqnarray*}
As expected, the interpretation of replication makes use of the $D$
operator defined above. Note that name allocation is intertwined with
prefix and new elimination.
\begin{eqnarray*}
  \meaningof{P}_3(n, p) 
    & := & 
      \meaningof{\prefix{n}{n'}{\prefix{p}{p'}{(\binpar{\meaningof{P}_2(n',p')}
        {(\binpar{D(x)}{\binpar{\outputp{n}{n'^l}}{\outputp{p}{p'^l}}})})}}}_4. \end{eqnarray*}
To handle prefix elimination we import most of Honda and Yoshida's
algorithm. The key difference is that we must allocate names that
guarantee freshness relative to the free names of the processes being
translated. In those rules below with a ``where'' clause, the specific
choice of combinators in the names is not so important as mentioning
those names and processes with respect to which the name mush be fresh.
For example, in rule $I$, for prefix to a parallel
composition, we must ensure that $v$ and $w$ are fresh with respect to
the names in $P$ and $Q$, and distinct from each other, and then we
must update both the name allocator for each parallel component and
the channels on which fresh names are communicate (so that there is no
interference between the two components) in the recursive call. 

We assume the following annotations ($+$ stands for the output and $-$
stands for the input), which denote how each name is used in the rules
of interaction, either as input, output, respectively:

\[\mathsf{m}(a^{+},v^{\pm});\mathsf{d}(a^{-},b^{+},c^{+});\mathsf{k}(a^{-});\mathsf{fw}(a^{-},b^{+});\mathsf{b}_{\mathsf{l}}(a^{-}b^{+});\mathsf{b}_{\mathsf{r}}(a^{-}b^{-});\mathsf{s}(a^{-}b^{-}c^{+})\]

As above the annotated polarities are preserved by reduction, e.g.
\[\binpar{\mathsf{d}(a^{-},b^{+},c^{+})}{\mathsf{m}(a^{+},v)} \to \binpar{\mathsf{m}(b^{+},v)}{\mathsf{m}(b^{+},v)}\]

Similarly, for economy of expression, we emulate Honda and Yoshida's
use of $\emph{c}$ to represent any combinator matching the arity
specification. It is worth pointing out that we use a slightly
different syntax for input-guarded processes. Where most readers
familiar with {\pic} are used to $x?(y)P$ we write $\mathsf{for}(y
\leftarrow x)P$, not only to be more instep with modern programming
languages, but also because it generalizes more naturally to join
constructions, such as $\mathsf{for}(y_1 \leftarrow x_1; \ldots; y_n
\leftarrow x_n)P$.
\[\small
\begin{array}{llrl}
(I)&  \meaningof{\prefix{p}{x}{\binpar{P}{Q}}}_4(n, q) 
    & := & 
    (\mathsf{d}(p,v,w)|\binpar{\meaningof{\prefix{v}{x}{P}}_{4}(n_{1}, q_{1})}{\meaningof{\prefix{w}{x}{Q}}_{4}(n_{2}, q_{2}) } \\
& & & \mbox{where $v = \quotep{(\mathsf{m}(q,\quotep{\binpar{\mathsf{b}_{\mathsf{l}}(q,n)}{\binpar{P}{Q}}}))}$,} \\
& & & \mbox{     $w = \quotep{(\mathsf{m}(q,\quotep{\binpar{\mathsf{b}_{\mathsf{r}}(q,n)}{\binpar{P}{Q}}}))}$,} \\
& & &\mbox{     $n_{1} = \quotep{(\binpar{\mathsf{b}_{\mathsf{l}}(v,w)}{\mathsf{m}(q,\quotep{\mathsf{m}(v,*w)})})}$,} \\
& & &\mbox{     $n_{2} = \quotep{(\binpar{\mathsf{b}_{\mathsf{r}}(v,w)}{\mathsf{m}(q,\quotep{\mathsf{m}(v,*w)})})}$,} \\
& & &\mbox{     $q_{1} = \quotep{(\binpar{\mathsf{b}_{\mathsf{l}}(n_{1},n_{2})}{\mathsf{m}(q,\quotep{\mathsf{m}(v,*w)})})}$,} \\
& & & \mbox{     $q_{2} = \quotep{(\binpar{\mathsf{b}_{\mathsf{r}}(n_{1},n_{2})}{\mathsf{m}(q,\quotep{\mathsf{m}(v,*w)})})}$ }\\
(V)&  \meaningof{\prefix{p}{x}{\emph{c}(v^{+},w)}}_4(n, q) 
    & := & 
    \mathsf{s}(p,a,v)|\emph{c}(a^{+},\vect{w})
    \mbox{, where $a = @(\binpar{\mathsf{m}(q,n)}{\emph{c}(v^{+},\vect{w})})$ and $x \not\in v : \vect{w}$}\\
(VI)&  \meaningof{\prefix{p}{x}{\emph{c}(v^{-},\vect{w})}}_4(n, q) 
    & := & 
    \mathsf{s}(p,v,a)|\emph{c}(a^{-},\vect{w})
    \mbox{, where $a = @(\mathsf{m}(q,n)|\emph{c}(v^{-},w))$ and $x \not\in v : \vect{w}$} \\
(VII)&  \meaningof{\prefix{p}{x}{\mathsf{m}(v,x)}}_4(n, q) 
    & := & 
    \mathsf{fw}(p,v) \\
(VIII)&  \meaningof{\prefix{p}{x}{\mathsf{fw}(x,v)}}_4(n, q) 
    & := & 
    \mathsf{b}_{\mathsf{l}}(p,v) \\
(IX)&  \meaningof{\prefix{p}{x}{\mathsf{fw}(v,x)}}_4(n, q) 
    & := & 
    \mathsf{b}_{\mathsf{r}}(p,v) \\
(X)&  \meaningof{\prefix{p}{x}{\emph{c}(\vect{v},x^{+},\vect{w})}}_4(n, q) 
    & := & 
    \meaningof{\prefix{p}{x}{\binpar{\mathsf{fw}(a,x)}{\emph{c}(\vect{v},a^{+},\vect{w})}}}_4(n', q)\\
    & & & \mbox{where $a = @(\binpar{\mathsf{m}(q,n)}{\emph{c}(\vect{v},x^{+},\vect{w})})$} \mbox{ and $n'= \quotep{(\mathsf{m}(a,\quotep{\mathsf{m}(q,*n)}))}$} \\
(XI)&  \meaningof{\prefix{p}{x}{\emph{c}(x^{-},\vect{v})}}_4(n, q) 
    & := & 
    \meaningof{\prefix{p}{x}{\binpar{\mathsf{fw}(x,a)}{\emph{c}(a^{-},\vect{v})}}}_4(n', q)\\
    & & & \mbox{where $a = @(\binpar{\mathsf{m}(q,n)}{\emph{c}(x^{-},v)})$} \mbox{ and $n'= \quotep{(\mathsf{m}(a,\quotep{\mathsf{m}(q,*n)}))}$} \\
(XII)&  \meaningof{\prefix{p}{x}{\mathsf{b}_{\mathsf{r}}(v,x^{-})}}_4(n, q) 
    & := & 
    \meaningof{\prefix{p}{x}{(\binpar{\mathsf{d}(v,w_{1},w_{2})}{\binpar{\mathsf{s}(w_{1},x,w_{3})}{\mathsf{b}_{\mathsf{r}}(w_{2},w_{3})}})}}_{4}(n', q) \\
    & & & \mbox{where $w_{1} = \quotep{(\binpar{\mathsf{b}_{\mathsf{l}}(q,n)}{\mathsf{m}(q,\quotep{\mathsf{b}_{\mathsf{r}}(v,x^{-})})})}$,} \\
    & & & \mbox{$w_{2} = \quotep{(\binpar{\mathsf{b}_{\mathsf{r}}(q,n)}{\mathsf{m}(q,\quotep{\mathsf{b}_{\mathsf{r}}(v,x^{-})})})}$,} \\
    & & & \mbox{$w_{3} = \quotep{(\binpar{\mathsf{b}_{\mathsf{l}}(p,v)}{\mathsf{m}(w_{1},w_{2}})}$, and } \\
    & & & \mbox{$n' = \quotep{(\mathsf{s}(w_{1},w_{2},w_{3}))}$} \\
(XIII)&  \meaningof{\prefix{p}{x}{\mathsf{s}(v,x^{-},w)}}_4(n, q) 
    & := & 
    \meaningof{\prefix{p}{x}{(\binpar{\mathsf{s}(v,w_{1},w_{2})}{\binpar{\mathsf{m}(w_{1},x)}{\mathsf{b}_{\mathsf{l}}(w_{2},w)}})}}_{4}(n', q) \\
    & & & \mbox{where $w_{1} = @(\binpar{\mathsf{b}_{\mathsf{l}}(q,n)}{\mathsf{s}(v,x^{-},w)})$, $w_{2} = @(\binpar{\mathsf{b}_{\mathsf{r}}(q,n)}{\mathsf{s}(v,x^{-},w)})$,} \\
    & & & \mbox{and $n' = \quotep{(\mathsf{m}(w_{1},w_{2}))}$}
\end{array}\]

\begin{example}
  To illustrate, let's look at the end-to-end translation of the {\pic} process
  \[\binpar{\prefix{x}{y}{0}}{(\mathsf{new}\;u)x!(u)}\].
  The {\ccomb} translation will yield (see Appendix~\ref{appendix:ccomb} for details)
\begin{eqnarray*}
  \meaningof{\binpar{\prefix{x}{y}{0}}{(\mathsf{new}\;u)x!(u)}} & = & \binpar{\meaningof{\prefix{x}{y}{0}}}{(\meaningof{\mathsf{new}\;u)x!(u)}} \\
  & = & \binpar{\meaningof{\prefixst{x}{y}{0}}}{(\mathsf{new}\;u)\meaningof{x!(u)}} \\
  & = & \binpar{\mathsf{k}(x)}{(\mathsf{new}\;u)\mathsf{m}(x,u)} \\
\end{eqnarray*}
If we new translate the resulting process into {\rhocc}, we get
\begin{eqnarray*}
  \meaningof{\binpar{\mathsf{k}(x)}{(\mathsf{new}\;u)\mathsf{m}(x,u)}} & = & \meaningof{\binpar{\mathsf{k}(x)}{(\mathsf{new}\;u)\mathsf{m}(x,u)}}_{2}(n,p)\\
  & = & \binpar{\meaningof{\mathsf{k}(x)}_{2}(n^{l},p^{l})}{\meaningof{\mathsf{(new}\;u)\mathsf{m}(x,u)}_{2}(n^{r},p^{r})} \\
  & = & \binpar{\mathsf{k}(x)}{\meaningof{\binpar{\prefix{p^{r}}{u}{\meaningof{\mathsf{m}(x,u)}_{2}(n^{rl},p^{rl})}}{\mathsf{m}(p^{r},n^{r})}}_{4}(n^{r},p^{r})} \\
  & = & \binpar{\mathsf{k}(x)}{\binpar{\meaningof{\prefix{p^{r}}{u}{\meaningof{\mathsf{m}(x,u)}_{2}(n^{rl},p^{rl})}}_{4}(n^{rl},p^{rl})}{\mathsf{m}(p^{r},n^{r})}} \\
  & = & \binpar{\mathsf{k}(x)}{\binpar{\meaningof{\prefix{p^{r}}{u}{\mathsf{m}(x,u)}}_{4}(n^{rl},p^{rl})}{\mathsf{m}(p^{r},n^{r})}} \\
  & = & \binpar{\mathsf{k}(x)}{\binpar{\mathsf{fw}(p^{r},x)}{\mathsf{m}(p^{r},n^{r})}} \\
\end{eqnarray*}
Note that all $\new$-binding is now interpreted, as in Wischik's
global $\pi$-calculus, as an input \cite{globalpi}: the
alpha-equivalence of $\new$ is handled by input-binding, while the
freshness is handled by providing a name that is guaranteed to be
fresh relative to the scope of use. Further, observe that to reduce
$\binpar{\mathsf{k}(x)}{(\mathsf{new}\;u)\mathsf{m}(x,u)}$ it is
necessary to invoke structural equivalence, yielding
$(\mathsf{new}\;u)\binpar{\mathsf{k}(x)}{\mathsf{m}(x,u)}$. This step
corresponds exactly to the forwarding of $n^{r}$ along $p^{r}$ to $x$ in the
reduction of
$\binpar{\mathsf{k}(x)}{\binpar{\mathsf{fw}(p^{r},x)}{\mathsf{m}(p^{r},n^{r})}}$.
\end{example}

It is also noteworthy that the translation is dependent on how the
parallel compositions in a process are associated. Different
associations will result in different bindings for $\new$
names. This will not result in different behavior, however:
while the {\rhocc} can encode different behaviors depending
on the choice of name, {\ccomb} cannot and the
embedding is insensitive to the choice.

\subsubsection{Faithfulness of the translation}
To establish the faithfulness of the translation we need to establish
that equivalence in {\ccomb} means equivalence in
{\rhocc}. Bisimulation is the natural notion of equivalence in both
calculi. However, as we observed before, names are \emph{global} in
the {\rhocc}, so we have to have a way of restricting to just the set
of names actually in play in order to get the notions of equivalence
to line up.

We define \emph{observation relation}, $\downarrow_{\mathcal{N}}$, over a set
of names, $\mathcal N$, as the smallest relation satisfying the rules:
(1) if $y \in {\mathcal N}, \; x \nameeq y$, then
$\mathsf{m}(x-) \downarrow_{\mathcal{N}} x$
and
(2) if 
$P\downarrow_{\mathcal{N}} x$ or $Q\downarrow_{\mathcal N} x$, 
then ${\binpar{P}{Q} \downarrow_{\mathcal{N}} x}$.

We write $P \Downarrow_{\mathcal{N}} x$ if there is $Q$ such that 
$P \wred Q$ and $Q \downarrow_{\mathcal{N}} x$.
The above definition is parametric in the the argument accepted in the
second position in the message combinator, $\mathsf{m}(x-)$, {\em i.e.} the payload
of the message: in the {\rhocc} the payload is a process,
while in {\ccomb} the payload is a name. Likewise, because the
definition of barbed bisimulation given below is dependent on the
definition of the observation relation, the definition is really a
template for the notion of bisimulation that must be instantiated to
the kind of payload accepted by the message combinator.

%% Notice that $\prefix{x}{y}{P}$ has no barb.  Indeed, in {\rhoc} as well
%% as other asynchronous calculi, an observer has no direct means to
%% detect if a message sent has been received or not.

\begin{definition}
%\label{def.bbisim}
An  ${\mathcal N}$-\emph{barbed bisimulation} over a set of names, ${\mathcal N}$, is a symmetric binary relation 
${\mathcal S}_{\mathcal N}$ between agents such that $P{\mathcal S}_{\mathcal N}Q$ implies:
(1) If $P \red P'$ then $Q \wred Q'$ and $P'{\mathcal S}_{\mathcal N}
Q'$; and 
 (2) If $P\downarrow_{\mathcal N} x$, then $Q\Downarrow_{\mathcal N} x$.
$P$ is ${\mathcal N}$-barbed bisimilar to $Q$, written
$P \wbbisim_{\mathcal N} Q$, if $P {\mathcal S}_{\mathcal N} Q$ for some ${\mathcal N}$-barbed bisimulation ${\mathcal S}_{\mathcal N}$.
\end{definition}

\begin{conjecture} 
  $P \wbbisim_{\pi} Q \iff \ldb P \rdb \wbbisim_{\texttt{FN}(P)} \ldb Q \rdb$.
\end{conjecture}

\begin{proof}[Proof sketch]
  The forward direction $\Rightarrow$ is immediate from the definition
  of the translation. The reverse direction is only interesting in the
  case of $!$ and $\mathsf{new}$. The replication case follows immediately
  from the calculation following Definition \ref{replication}. 
  In the $\new$ case, transitions on $\mathsf{new}$-bound
  names will be in one-to-one correspondence with names provided by
  the name parameters of the translation function. By construction,
  these are not observable by the observation relation.
  We shall submit the full proof to the special issue of
  PLACES'20 (journal) if this paper is accepted.
\end{proof}

\begin{remark}
  In light of this theorem, it is worth pointing out that this version
  of the {\rhocc} has no rule for \emph{introducing} terms of
  the form $\dropn{x}$. The $b_r$ and $b_l$ combinators introduce new
  names from processes, but the do not introduce new reflection
  terms. Yet, this calculus suffices to faithfully represent the
  {\ccomb}. This is because the translation function is
  carefully introducing just those terms, via the $D(x,v,w)$ operator,
  guided by the use of replication in the source to the
  translation. 
\end{remark}

\section{Conclusion, related and future work}
We have shown how to construct a concurrent higher-order combinator
calculus that uses reflection to avoid the necessity for new and bound
names.  The {\ccomb}, and therefore the asynchronous {\pic}, have a
faithful embedding into the calculus.

Apart from the work in \cite{DBLP:conf/popl/HondaY94,DBLP:journals/tcs/Yoshida02},
the work \cite{DBLP:journals/toplas/RajaS97}
has achieved a similar goal with ours, but
without reflection, using a technique pioneered by Quine
\cite{Quine59,Quine60}, who was also a pioneer of reflective
techniques. The key difference with respect to the treatment of fresh
names is that they introduce globally unique names and then
internalize a process that generates them. The approach taken here is
more local, in the sense that reflection allows for the generation of
names fresh with respect to a specific scope. This means there is no
need for global uniqueness nor a centralized resource for generating
the fresh names.

While the results in this paper are interesting in their own right we
developed them to serve a larger purpose. In a forthcoming paper we
demonstrate an algorithm taking a graph-enriched Lawvere theory
(representing a formal specification of a term calculus) and a more
vanilla Lawvere theory (representing a specification of a notion of
collection, such as set, or bag, or list, etc) and produce a type
system for the term calculus enjoying soundness and
completeness. Nominal phenomena do not neatly fit inside the
expressive power of graph-enriched Lawvere theories, thus potentially
limiting the scope of the applicability of this algorithm. However,
with the reflective techniques we can extend this algorithm to cover
many languages and term calculi with binding; thus, providing a
strongly typed version of a reflective, mobile concurrent
calculus. Typing of this kind is especially important in environments
like smart contracting blockchains where a single race condition can cost the
whole network tens of millions of dollars.

\paragraph{Acknowledgments} We would like to acknowledge Nobuko Yoshida for her thoughtful feedback and comments. They made the paper much better.

\bibliographystyle{eptcs} \bibliography{rhocomb}

\section{Appendix: the {\ccomb} revisited} \label{appendix:ccomb}
We summarize the original {\ccomb} \cite{DBLP:journals/tcs/Yoshida02} below.

\begin{mathpar}
  \inferrule* [lab=atom] {} { P \bc 0 \;|\; \mathsf{m}(a,b) \;|\; \mathsf{d}(a,b,c) \;|\; \mathsf{k}(a) \;|\; \mathsf{fw}(a,b) \;|\; \mathsf{b}_{\mathsf{r}}(a,b) \;|\; \mathsf{b}_{\mathsf{l}}(a,b) \;|\; \mathsf{s}(a,b,c) }
  \and
  \inferrule* [lab=process] {} {\bm \; (\new\; a)P \;|\; P|P \;|\; \mathsf{*}P}
\end{mathpar}

We write $\mathsf{c};\mathsf{c'};\ldots;$ to denote these
agents. $\mathsf{m}(a,b)$ (message) carries $a$ name $b$ to name $a$,
$\mathsf{d}$ (duplicator) distributes a message to two locations,
$\mathsf{fw}$ (forwarder) forwards a message (thus linking two
locations), $\mathsf{k}$ (killer) kills a message, while
$\mathsf{b}_{\mathsf{r}}$ (right binder), $\mathsf{b}_{\mathsf{l}}$
(left binder) and $\mathsf{s}$ (synchroniser) generate new links. In
particular $\mathsf{b}_{\mathsf{r}}$ and $\mathsf{b}_{\mathsf{l}}$
represent two different ways of binding names – in
$\mathsf{b}_{\mathsf{r}}$ one uses the received name for output, while
in $\mathsf{b}_{\mathsf{l}}$ one uses it for input. In contrast,
$\mathsf{s}$ is used for pure synchronisation without value passing,
which is indeed necessary in interaction scenarios.

As in the {\pic}, the $\new$ operator is a binding operator
for names, so {\ccomb} also has a notion of free and bound names.

\begin{mathpar}
  \freenames{\pzero} := \emptyset
  \and
  \freenames{\mathsf{k}(a)} := \{ a \}
  \and
  \freenames{\mathsf{m}(a,b)} = \freenames{f(a,b)} = \freenames{\mathsf{b}_{\mathsf{r}}(a,b)} = \freenames{\mathsf{b}_{\mathsf{l}}(a,b)} := \{ a, b \}
  \and
  \freenames{\mathsf{d}(a,b,c)} = \freenames{\mathsf{s}(a,b,c)} := \{ a, b, c \}
  \and
  \freenames{(\new\;a)P} := \freenames{P} \setminus \{ a \}
  \and
  \freenames{P|Q} := \freenames{P} \cup \freenames{Q}
  \and
  \freenames{!P} := \freenames{P}
\end{mathpar}

The bound names of a process, $\boundnames{P}$, are those names occurring in $P$
that are not free. For example, in $(\new\; b)\mathsf{m}(a,b)$, the name $a$ is free, while $b$ is bound.

In the following definition, $\vect{x}$ indicates a list of names,
$u:\vect{x}$ indicates the concatenation of $u$ onto the vector, and
abuse set notation $u \in \vect{x}$ to assert or require that $u$
occurs in $\vect{x}$.

\begin{definition}
Two processes, $P,Q$, are alpha-equivalent if $P = Q\{\vect{y}/\vect{x}\}$ for
some $\vect{x} \in \boundnames{Q},\vect{y} \in \boundnames{P}$, where $Q\{\vect{y}/\vect{x}\}$
denotes the capture-avoiding substitution of $\vect{y}$ for $\vect{x}$ in $Q$.
\end{definition}

\begin{definition}
  The {\em structural congruence} $\equiv$
  between processes \cite{SangiorgiWalker} is the least congruence containing
  alpha-equivalence and satisfying the commutative monoid laws
  (associativity, commutativity and $\pzero$ as identity) for parallel
  composition $|$.
\end{definition}

\noindent Rewrite rules
\[\begin{array}{rl}
  \mathsf{d}(a,b,c) | \mathsf{m}(a,x) & \red \mathsf{m}(b,x) | \mathsf{m}(c,x) \\
  \mathsf{k}(a) | \mathsf{m}(a,x) & \red 0 \\
  \mathsf{fw}(a,b) | \mathsf{m}(a,x) & \red \mathsf{m}(b,x) \\
\end{array} \quad \quad
\begin{array}{rl}
  \mathsf{b}_{\mathsf{r}}(a,b) | \mathsf{m}(a,x) & \red \mathsf{fw}(b,x) \\
  \mathsf{b}_{\mathsf{l}}(a,b) | \mathsf{m}(a,x) & \red \mathsf{fw}(x,b) \\
  \mathsf{s}(a,b,c) | \mathsf{m}(a,x) & \red \mathsf{fw}(b,c)
\end{array}\]
\[\begin{array}{rl}
  \mathsf{*}P & \red P|\mathsf{*}P \\
\end{array}\]
\begin{mathpar}
  \inferrule* {{P} \red {P}'} {{{P} | {Q}} \red {{P}' | {Q}}}
  \and
  \inferrule* {{{P} \scong {P}'} \andalso {{P}' \red {Q}'} \andalso {{Q}' \scong {Q}}}{{P} \red {Q}}
\end{mathpar}

\subsubsection{Translating the {\pic} into {\ccomb}}
We assume the following annotations ($+$ stands for the output and $-$
stands for the input), which denote how each name is used in the rules
of interaction:

\[\mathsf{m}(a^{+},v^{\pm});\mathsf{d}(a^{-},b^{+},c^{+});\mathsf{k}(a^{-});\mathsf{fw}(a^{-},b^{+});\mathsf{b}_{\mathsf{l}}(a^{-}b^{+});\mathsf{b}_{\mathsf{r}}(a^{-}b^{-});\mathsf{s}(a^{-}b^{-}c^{+})\]

Note the annotated polarities are preserved by reduction, e.g.
\[\binpar{\mathsf{d}(a^{-},b^{+},c^{+})}{\mathsf{m}(a^{+},v)} \to \binpar{\mathsf{m}(b^{+},v)}{\mathsf{m}(b^{+},v)}\]

It is worth pointing out that we use a slightly different syntax for input-guarded processes. Where most readers familiar with $x?(y)P$ we write $\mathsf{for}(y \leftarrow x)P$, not only to be more instep with modern programming languages, but also because it generalizes more naturally to join constructions, such as $\mathsf{for}(y_1 \leftarrow x_1; \ldots; y_n \leftarrow x_n)P$.

\begin{mathpar}
  \meaningof{a!(b)} = \mathsf{m}(a,b)
  \and
  \meaningof{\mathsf{for}(x \leftarrow a)P} = \mathsf{for}^*(x \leftarrow a)\meaningof{P}
  \and
  \meaningof{P|Q} = \meaningof{P}|\meaningof{Q}
  \and
  \meaningof{(\mathsf{new}\;x)P} = (\mathsf{new}\;x)\meaningof{P}
  \and
  \meaningof{\mathsf{*}P} = \mathsf{*}\meaningof{P}
  \and
  \meaningof{0} = 0
\end{mathpar}

where

\[\begin{array}{llrl}
(I)&\mathsf{for}^*(x \leftarrow a)(P|Q) &=& (\mathsf{new}\; c_1c_2)(\mathsf{d}(a,c_1,c_2) | \mathsf{for}^*(x \leftarrow c_1)P | \mathsf{for}^*(x \leftarrow c_2)Q)\\
(II)&\mathsf{for}^*(x \leftarrow a)(\mathsf{new}\; c')P &=& (\mathsf{new}\; c)\mathsf{for}^*(x \leftarrow a)P\substn{c}{c'} \\
(III)& \mathsf{for}^*(x \leftarrow a)\pzero &=& \mathsf{k}(a) \\
(IV)&\mathsf{for}^*(x \leftarrow a)\mathsf{*}P &=& (\mathsf{new}\;c)(\mathsf{fw}(a,c) | \mathsf{*}\mathsf{for}^*(x \leftarrow c)(P | \mathsf{m}(c,x)))\\
(V)&\mathsf{for}^*(x \leftarrow a)c(v^+,\vec{w}) &=& (\mathsf{new}\;c)(\mathsf{s}(a,c,v) | c(c^+,\vec{w})) \mbox{\quad $x \notin \{v\vec{w}\}$}\\
(VI)&\mathsf{for}^*(x \leftarrow a)c(v^-,\vec{w})&=& (\mathsf{new}\;c)(\mathsf{s}(a,v,c) | c(c^-,\vec{w})) \mbox{\quad $x \notin \{v\vec{w}\}$}\\
(VII)&\mathsf{for}^*(x \leftarrow a)\mathsf{m}(v,x)&=& \mathsf{fw}(a,v) \mbox{\quad $x \neq v$}\\
(VIII)&\mathsf{for}^*(x \leftarrow a)\mathsf{fw}(x,v)&=& \mathsf{b}_{\mathsf{l}}(a,v) \mbox{\quad $x \neq v$}\\
(IX)&\mathsf{for}^*(x \leftarrow a)\mathsf{fw}(v,x)&=& \mathsf{b}_{\mathsf{r}}(a,v) \mbox{\quad $x \neq v$}\\
(X)&\mathsf{for}^*(x \leftarrow a)c(\vec{v}_1x^+\vec{v}_2)&=&(\mathsf{new}\; c)\mathsf{for}^*(x \leftarrow a)(\mathsf{fw}(c,x) | c(\vec{v}_1c\vec{v}_2)) \mbox{\quad $x \notin \vec{v}_1$}\\
(XI)&\mathsf{for}^*(x \leftarrow a)c(x^-\vec{v})&=& (\mathsf{new}\; c)\mathsf{for}^*(x \leftarrow a)(\mathsf{fw}(x,c) | c(c\vec{v}))\\
(XII)&\mathsf{for}^*(x \leftarrow a)\mathsf{b}_{\mathsf{r}}(v,x^-)&=& (\mathsf{new}\; c_1c_2c_3)\mathsf{for}^*(x \leftarrow a)(\mathsf{d}(v,c_1,c_2)|\mathsf{s}(c_1,x,c_3)|\mathsf{b}_{\mathsf{r}}(c_2,c_3)) \mbox{\quad $x \notin \{v \}$}\\
(XIII)&\mathsf{for}^*(x \leftarrow a)\mathsf{s}(v,x^-,w) &=& (\mathsf{new}\; c_1c_2)\mathsf{for}^*(x \leftarrow a)(\mathsf{s}(v,c_1,c_2) | \mathsf{m}(c_1,x) | \mathsf{b}_{\mathsf{l}}(c_2,w)) \mbox{\quad $x \notin \{v \}$}
\end{array}\]

\section{Appendix: Translating the {\pic} into the {\rhoc}} \label{appendix:pic2rhoc}
\subsection{\rhoc}

It is striking to compare the {\rhoc} with the {\pic} as the former is
a vastly simpler theory and yet has enjoys more
features. Specifically, the specification of the {\rhoc}'s structural
equivalence and reduction rules are notably simpler, with one small
technical caveat: name equivalence depends on structural equivalence
which depends upon $\alpha$-equivalence which depends on name
equivalence. Meredith and Radestock show that this cycle terminates
innocently due to the design of the grammar. This technical complexity
seems a small price to pay for a much simpler calculus that enjoys
both higher order communication as well reflection and meta-programming
features.

\begin{mathpar}
\inferrule* [lab=process] {} {P, Q \bc \pzero \;\bm\; \mathsf{for}( y
  \leftarrow x )P \;\bm\; x!(Q) \;\bm\;	\mathsf{*}x \;\bm\; P|Q }
\and
\inferrule* [lab=name] {} {x, y \bc \mathsf{@}P }
\end{mathpar}

\begin{definition}
\emph{Free and bound names} The calculation of the free names of a
process, $P$, denoted $\freenames{P}$ is given recursively by

\begin{mathpar}
  \freenames{\pzero} = \emptyset
  \and
  \freenames{\mathsf{for}(y \leftarrow x)(P)} = \{ x \} \cup \freenames{P}\setminus\{y\}
  \and
  \freenames{x!(P)} = \{ x \} \cup \freenames{P}
  \and
  \freenames{P|Q} = \freenames{P} \cup \freenames{Q}
  \and
  \freenames{\mathsf{}{x}} = \{ x \} \\
\end{mathpar}

An occurrence of $x$ in a process $P$ is \textit{bound} if it is not
free. The set of names occurring in a process (bound or free) is
denoted by $\names{P}$.
\end{definition}

\begin{definition}
  The {\em structural congruence} $\equiv$
  between processes \cite{SangiorgiWalker} is the least congruence containing
  alpha-equivalence and satisfying the commutative monoid laws
  (associativity, commutativity and $\pzero$ as identity) for parallel
  composition $|$.
\end{definition}

\begin{definition}
  The {\em name equivalence} $\nameeq$ is the least congruence
  satisfying these equations
  \begin{mathpar}
  \inferrule*[lab=Quote-drop] {}{ \quotep{\dropn{x}} \nameeq x }
  \and
  \inferrule*[lab=Struct-equiv] { P \scong Q } { \quotep{P} \nameeq \quotep{Q} }
  \end{mathpar}
\end{definition}

\subsection{Operational semantics} 

\begin{mathpar}
  \inferrule* [lab=COMM] {x_{trgt} \nameeq x_{src}} {\mathsf{for}( y \leftarrow x_{trgt} )P | x_{src}!(Q)
  \red P\substn{\mathsf{@}Q}{y}}
  \and
  \inferrule* [lab=PAR]{P \red P'}{P|Q \red P'|Q}
  \and
  \inferrule* [lab=EQUIV]{{P \scong P'} \andalso {P' \red Q'} \andalso {Q' \scong Q}}{P \red Q}
\end{mathpar}

\subsection{\pic}

In this presentation of the {\pic} we update the syntax for
input-guarded processes to reflect the widespread adoption of
comprehension notation in languages ranging from Scala to Python for
use in reactive programming. Here we go with the Scala notation
writing $\mathsf{for}( y \leftarrow x )P$ where Milner might have
written $x?(y)P$. Admittedly, it's somewhat more verbose, but conveys
to a younger generation of programmers more familiar with reactive
programming the intended semantics of the expression. Similarly, since
we keep the output expression $x!(y)$ we avoid collision with the
traditional notation for reflection ($!P$) by writing $\mathsf{*}P$
instead, which is at least somewhat reminiscent of the Kleene star and
the familiar from regular expressions.

\begin{mathpar}
\inferrule* [lab=process] {} {P, Q \bc \pzero \;\bm\; \mathsf{for}( y
  \leftarrow x )P \;\bm\; x!(y) \;\bm\; (\mathsf{new}\;x)P \;\bm\; P|Q \;\bm\;	\mathsf{*}P}
\end{mathpar}

Note that there is no production for $x$'s and $y$'s in the
grammar. This reflects the fact that the {\pic} is \emph{parametric}
in the collection channel names. That collection merely has to be
countably infinite and have an effective equality. As such it is
perfectly reasonable to choose a collection names, namely the names of
the {\rhoc}. We can make this choice without loss of generality
because we can always choose some other countably infinite set with an
effective equality, say $\mathcal{N}$ and then require an invertible
map, $\mathsf{code} : \mathcal{N} \to @\mathsf{Proc}$.

\subsection{Structural congruence}

\begin{definition}
The {\em structural congruence}, $\equiv$, between processes is 
the least congruence closed with respect to
alpha-renaming, satisfying the abelian monoid laws for 
parallel (associativity, commutativity and $\pzero$ 
as identity), and the following axioms:
\begin{enumerate}
\item the scope laws:
\begin{eqnarray*}
 (\mathsf{new}\;x)\pzero & \equiv & \pzero, \\
 (\mathsf{new}\;x)(\mathsf{new}\;x)P & \equiv & (\mathsf{new}\;x)P, \\
 (\mathsf{new}\;x)(\mathsf{new}\;y)P & \equiv & (\mathsf{new}\;y)(\mathsf{new}\;x)P, \\
 P|(\mathsf{new}\;x)Q & \equiv & (\mathsf{new}\;x)P|Q, \; \mbox{\textit{if} }x \not\in \freenames{P} 
\end{eqnarray*}
\item
the recursion law:
\begin{eqnarray*}
 \mathsf{*}P \equiv P|\mathsf{*}P
\end{eqnarray*}
\item
the name equivalence law:
\begin{eqnarray*}
 P \equiv P \substn{x}{y}, \; \mbox{\textit{if} }x \nameeq y
\end{eqnarray*}
\end{enumerate}
\end{definition}

\subsection{Operational semantics} 

\begin{mathpar}
  \inferrule* [lab=COMM] {} {\mathsf{for}( y \leftarrow x )P | x!(v) \red P\substn{v}{y}}
  \and
  \inferrule* [lab=PAR]{P \red P'}{P|Q \red P'|Q}
  \and
  \inferrule* [lab=NEW]{P \red P'}{(\mathsf{new}\;x)P \red (\mathsf{new}\;x)P'}
  \and
  \inferrule* [lab=EQUIV]{{P \scong P'} \andalso {P' \red Q'} \andalso {Q' \scong Q}}{P \red Q}
\end{mathpar}


Here we stick with tradition and write $\wred$ for $\red^*$, and rely
on context to distinguish when $\red$ means reduction in the {\pic}
and when it means reduction in the {\rhoc}. The set of {\pic}
processes will be denoted by $\Proc_{\pi}$.

\subsection{The translation}

The translation will be given by a function, $\meaningof{-}( -, - ) :
\Proc_{\pi} \times \QProc \times \QProc \red \Proc$. The guiding
intuition is that we construct alongside the process a distributed memory
allocator, the process' access to which is mediated through the second argument
to the function. The first argument determines the shape of the memory
for the given allocator.

Given a process, $P$, we pick $n$ and $p$ such that $n \neq p$ and
distinct from the free names of $P$. For example, $n = \quotep{\Pi_{m
\in \freenames{P}}\outputp{m}{\quotep{\pzero}}}$ and $p =
\quotep{\Pi_{m \in
\freenames{P}}\mathsf{for}({\quotep{\pzero}} \leftarrow {m}){\pzero}}$. Then

\begin{equation*}
	\meaningof{P} = \meaningof{P}_{2nd}( n, p )
\end{equation*}

where

\begin{eqnarray*}
   	\meaningof{\pzero}_{2nd} (n,p) & = & \pzero \\
   	\meaningof{x!(@Q)}_{2nd} (n,p) & = & x!(Q) \\
   	\meaningof{\mathsf{for}( y \leftarrow x) P}_{2nd} (n,p) 
   		& = & 
 		\mathsf{for}( y \leftarrow x ) \meaningof{P}_{2nd} (n,p) \\
   	\meaningof{P | Q}_{2nd} (n,p) 
   		& = & 
 		\meaningof{P}_{2nd} (n^{l},p^{l}) | \meaningof{Q}_{2nd} (n^{r},p^{r}) \\
%    	\meaningof{\id{!} P}_{2nd} (  n, p )
%    		& = & \binpar{\lift{x}{\binpar{upn( n^{lr}, p^{lr}, n^{rl}, p^{rl} )}
% 						      {\meaningof{P}_{3rd}( n^{lr}, p^{lr}, n^{rl}, p^{rl} )}}}
% 		             {\binpar{D(x)}{\binpar{\outputp{n^{lr}}{n}}{\outputp{p^{lr}}{p}}}} \\
   	\meaningof{\mathsf{*} P}_{2nd} (n,p)
   		& = & x!(\meaningof{P}_{3rd}( n^{r}, p^{r} ))|D(x)|n^{r}!n^{l}|p^{r}!p^{l} \\
   	\meaningof{(\mathsf{new} \; x) P}_{2nd} (n,p) 
   		& = & 
 		\mathsf{for}(x \leftarrow p)\meaningof{P}_{2nd} ( n^{l}, p^{l} )|p!(n) \\
\end{eqnarray*}

and

\begin{eqnarray*}
	x^{l} & \triangleq & \quotep{\outputp{x}{x}} \\
	x^{r} & \triangleq & \quotep{\prefix{x}{x}{\pzero}} \\
	\meaningof{P}_{3rd}( n'', p'' ) 
		& \triangleq & 
			\prefix{n''}{n}{\prefix{p''}{p}{(\binpar{\meaningof{P}_{2nd}(  n, p )}
							        {(\binpar{D(x)}{\binpar{\outputp{n''}{n^{l}}}{\outputp{p''}{p^{l}}}})})}} \\
\end{eqnarray*}

\begin{remark}
	Note that all $\mathsf{new}$-binding is now interpreted, as in Wischik's
	global $\pi$-calculus, as an input guard \cite{globalpi}.
\end{remark}
	
\begin{remark}
	It is also noteworthy that the translation is dependent on how
	the parallel compositions in a process are
	associated. Different associations will result in different
	bindings for $\mathsf{new}$-ed names. This will not result in different
	behavior, however, as the bindings will be consistent
	throughout the translation of the process.
\end{remark}

\begin{theorem}[Correctness]	
	$P \wbbisim_{\pi} Q \iff \ldb P \rdb \wbbisim_{r(\texttt{FN}(P))} \ldb Q \rdb$.
\end{theorem}

\emph{Proof sketch}: An easy structural induction.

One key point in the proof is that there are contexts in the {\rhoc}
that will distinguish the translations. But, these are contexts that
can see the fresh names, $n$, and the communication channel, $p$, for
the `memory allocator'. These contexts do not correspond to any
observation that can be made in the {\pic} and so we exclude them in
the {\rhoc} side of our translation by our choice of ${\mathsf N}$
for the bisimulation. This is one of the technical motivations behind
our introduction of a less standard bisimulation.

\begin{example}
	In a similar vein consider, for an appropriately chosen $p$ and $n$ we have
	\begin{equation*}
		\meaningof{(\mathsf{new}\;v)(\mathsf{new}\;v) u!(v)} = \mathsf{for}(v \leftarrow p)(\mathsf{for}({v} \leftarrow {\quotep{p!(p)}})(u!(v)|\quotep{p!(p)})!(\quotep{n!(n)})) | p!(n)
	\end{equation*}
	and
	\begin{equation*}
		\meaningof{(\mathsf{new}\;v)u!(v)} = \mathsf{for}(v \leftarrow p)(u!(v) )|p!(n)
	\end{equation*}

	Both programs will ultimately result in an output of a single
	fresh name on the channel $u$. But, the former program will
	consume more resources. Two names will be allocated; two memory
	requests will be fulfilled. The {\rhoc} can see this, while the
	{\pic} cannot. In particular, the {\pic} requires that
	$(\mathsf{new}\;x)(\mathsf{new}\;x)P \equiv (\mathsf{new} x)P$.

	Implementations of the {\pic}, however, having the property that
	$(\mathsf{new}\;x)P$ involves the allocation of memory for the
	structure representing the channel $x$ come to grips with the
	implications this requirement has regarding memory management. If
	memory is allocated upon encountering the $\mathsf{new}$-scope, there are
	situations where the left-hand side of the equation above will
	fail while the right-hand will succeed. Remaining faithful to the
	equation above requires that such implementations are
	\textit{lazy} in their interpretation of $(\mathsf{new}\;x)P$, only
	allocating the memory for the fresh channel at the first moment
	when that channel is used.

	Having a detailed account of the structure of names elucidates
	this issue at the theoretical level and may make way to offer
	guidance to implementations.
\end{example}

\end{document}

